% Jak psát vzorová řešení
%
% překlad pomocí xelatex -shell-escape -8bit jakpsat.tex
% - potřebuje minted.sty v2.0-alpha2
% - potřebuje python-pygments

\documentclass[fykos]{fksbase}
\setcounter{year}{27}

\geometry{ignoreheadfoot,a4paper,margin=1.5cm}
\fancyhf{}
\renewcommand{\headrulewidth}{0pt}

\usepackage{multicol}
\usepackage{colortbl}

\usepackage{minted}
\definecolor{mintedbg}{rgb}{0.90, 0.90, 1}
\newminted[texminted]{tex}{bgcolor=mintedbg, tabsize=2,
	xleftmargin=1mm, xrightmargin=1mm}
\newmintedfile[texfile]{tex}{bgcolor=mintedbg, tabsize=2,
	xleftmargin=1mm, xrightmargin=1mm,obeytabs=true}
\newmint{tex}{showspaces=false}
\newmintinline{tex}{showspaces=false}

\usepackage{mdframed}
\mdfsetup{backgroundcolor=blue!2, linecolor=blue!30,
	linewidth=3pt, leftline=false, rightline=false,
	innertopmargin=0, innerbottommargin=0, skipabove=15pt, skipbelow=15pt,}

\newcolumntype{T}{>{\columncolor{mintedbg}}m{0.3\textwidth}}
\newcolumntype{V}{>{\columncolor{blue!2}}m{0.2\textwidth}}
\newcolumntype{N}{>{\columncolor{white}}m{0.468\textwidth}}

\newenvironment{texcode}
	{\VerbatimEnvironment\hskip-10pt
		 \minipage[t]{0.52\textwidth}\vspace{0pt}\begin{texminted}{tex}}
	{\end{texminted} \endminipage}
\newenvironment{texview}
	{\hskip 4pt \minipage[t]{0.50\textwidth}\vskip 4pt \vspace{0pt}\parindent=1em}
	{\vskip 6pt \endminipage}
\newenvironment{texexamplex}
	{\begin{mdframed}}
	{\vskip -2pt\end{mdframed}}
\newenvironment{texsource}
	{\VerbatimEnvironment\begin{mdframed}[backgroundcolor=mintedbg]%
		\hskip-16pt\begin{texminted}{tex}}
	{\end{texminted}\hskip-40pt\end{mdframed}}
\newenvironment{texexample}
	{\VerbatimEnvironment\VerbatimOut[tabsize=2]{\jobname.tmp}}
	{\endVerbatimOut\begin{mdframed}\hskip-10pt\minipage[t]{0.52\textwidth}%
		\vspace{0pt}\texfile{\jobname.tmp}\endminipage%
		\hskip4pt\minipage[t]{0.50\textwidth}\vskip 4pt%
		\vspace{0pt}\parindent=1em\input{\jobname.tmp}%
		\vskip6pt\endminipage\vskip-2pt\end{mdframed}}
\newenvironment{texexampletab}
	{\begin{mdframed}\hskip-10pt\arrayrulecolor{blue!25}\begin{tabular}{TVN}}
	{\end{tabular}\arrayrulecolor{black}\end{mdframed}}

\begin{document}

\begin{center}
	\includegraphics{logo2.eps}
	\hspace{10pt}\raisebox{0.6cm}{\normalfont\sffamily\slshape\Huge
		Jak psát vzorová řešení}
	\vskip0.6cm
\end{center}

Tento text je určen zejména organizátorům FYKOSu/Výfuku a~má pomoci při psaní
vzorových řešení. Většinu zde popsaných příkazů však lze použít i~v~jiných
dokumentech a~usnadnit si tak práci hlavně při sazbě matematiky. V~tom případě
do hlavičky souboru přidejte balík \texttt{fkssugar}
pomocí~{\texinline|\usepackage{fkssugar}|}.

\section{Soubory s~úlohami}

Soubory s~úlohami se nachází v~adresáři~\texttt{problems}, jsou pojmenované
\texttt{problemS-U.tex}. Nové soubory s~úlohami nepřidávejte, generuje je TeXař
skriptem.

\begin{texsource}
\probbatch{2}
\probno{6}
\probname{Temelínská}
\proborigin{Karel přemýšlel nad ČEZem.}
\probpoints{4}
\probauthors{kolar}
\probsolauthors{grund}
\probsolvers{51}
\probavg{2.32}
\probtask{%
	Odhadněte, kolik jaderného paliva se spotřebuje v~jaderné elektrárně\dots}
\probsolution{%
}
\end{texsource}

Obsah souboru~\texttt{problemS-U.tex}:
\begin{compactitem}
	\item \texinline|\probbatch|~--~série úlohy
	\item \texinline|\probno|~--~číslo úlohy
	\item \texinline|\probname|~--~název úlohy (začínající malým písmenem, nejde-li
		o~vlastní jméno)
	\item \texinline|\proborigin|~--~původ úlohy (krátká věta, zobrazí se v~brožurce pod
		zadáním)
	\item \texinline|\probauthors|~--~autor úlohy, použije se při sazbě ročenky. Jméno
	je většinou ve formátu příjmení malými písmeny bez diakritiky (viz soubor
	\texttt{tex/latex/fykosx/fksemails.tex} v~repozitáři \texttt{texmf}). Více jmen
	(nejvýše~3) se odděluje čárkou bez mezery.
	\item \texinline|\probsolauthors|~--~autor vzorového řešení (ve stejném formátu),
		vypíše se pod vzorovým řešením (kromě ročenky, kde jsou autoři uvedeni na konci)
	\item \texinline|\probsolvers|~--~počet řešitelů (generuje se automaticky
		z~výsledkovek)
	\item \texinline|\probavg|~--~průměrný počet bodů (generuje se automaticky
		z~výsledkovek)
	\item \texinline|\probtask|~--~zadání úlohy. Ihned za posledním znakem musí
	být ukončena závorka.
	\item \texinline|\probsolution|~--~vzorové řešení úlohy. Toto by mělo být jediné
		pole, které upravuje autor vzorového řešení. Ihned za posledním znakem musí být
		ukončena závorka.
\end{compactitem}

\section{Překlad}

Překládáme pomocí \texttt{make}. Pokud nemůžete výchozí variantu
použít\footnotei{,}{I na Windows je možné tuto variantu bez problému provozovat
a~mít tak možnost si nechat automaticky přeložit celý ročník včetně obrázků,
	aktuálních výsledkovek apod., nicméně to vyžaduje složitější nastavení. Autorům
	vzorových řešení by měl naprosto dostačovat druhý popisovaný způsob.} vytvořte
soubor \texttt{solution.tex} s~obsahem uvedeným níže. Změňte cestu k~souboru
úlohy, popř.~název semináře a~ročník a~přeložte příkazem \texttt{xelatex}. Vždy
překládejte soubor~\texttt{solution.tex}, nikoliv
soubor~\texttt{problems/problemS-U.tex}\footnote{Používate-li např.~TeXstudio,
	otevřete soubor~\texttt{solution.tex} a~zvolte Volby~→~Prohlásit současný
	dokument za Hlavní dokument. Poté můžete mít otevřený soubor s~úlohou, ale
	překládat se vždy bude soubor~\texttt{solution.tex}}. Obrázky nebudou do souboru
vloženy, protože v~některých případech by překlad tímto způsobem kvůli
chybějícím obrázkům selhal.

\begin{texsource}
\documentclass[fykos]{fkssolution} % nahradit jméno semináře
\setcounter{year}{27} % nahradit ročník
\renewcommand{\includegraphics}[2][]{\fbox{#2}}
\renewcommand\plotfig[4][h]{\begin{figure}[#1]\centering\fbox{#2}\caption{#3}\label{#4}\end{figure}}
\input{"fykos27/problems/problem1-7"} % nahradit soubor s úlohou
\begin{document}
	\problemsolutionsingle
	\makefooter
\end{document}
\end{texsource}

\section{Obecné zásady}

Řešení pište bez chyb! Kdo tam bude mít chyby, bude zmrskán!

Všechny texty týkající se FYKOSu jsou psány v~typografickém systému \TeX
(využívajíce {\LaTeX}ových maker). Řešení úloh pište v~libovolném textovém
editoru (např.~\texttt{TeXstudio}\footnotei{,}{\url{http://texstudio.sourceforge.net},
	dostupné pro Windows, Linux i~Mac OS X} které vám s~psaním pomocí různých
klikátek velmi pomůže) do souboru úlohy v~kódování UTF-8.

Výsledný soubor uložte do repozitáře. V~případě nouze nebo nemáte-li do
repozitáře přístup pošlete e-mailem TeXaři (v~tom případě je nutné je poslat
s~dostatečným předstihem před uzávěrkou), přičemž připojte alespoň dva odstavce
omluveného textu, proč jste soubor nevložili do repozitáře sami. Do repozitáře
nevkládejte soubory, které nejdou přeložit, a~v~žádném případě nevkládejte nebo
neposílejte soubory v~jiném kódování než UTF-8.

Pokud nemáte přístup do repozitáře a~chcete provést úpravu v~řešení, stáhněte si
aktuální verzi z~Astrid\footnotei{,}{\url{http://astrid.fykos.cz}, uživatel
	\texttt{fykos}, heslo obligátní} upravte ji a~co nejdříve pošlete e-mailem
TeXaři, aby se omezily konflikty úprav. Kdo pošle opravený původní soubor, který
od té doby byl změněn, bude vyzván k~opětovné úpravě správné verze souboru.

Text pište česky s~háčky a~čárkami v~první osobě množného čísla, asi
po~60~znacích (po~logických celcích) se snažte odřádkovávat. Více mezer
a~(jednotlivé) odřádkování nemá vliv na vizuální podobu výsledného textu.

Text pište v~první osobě množného čísla, na konci svého řešení můžete uvést
hlavní chyby, kterých se řešitelé dopouštěli (ty jsou součástí brožurky, nicméně nesází
se do ročenky). Vzorové řešení by mělo končit větou, nikoliv vzorcem.

\section{Základy \LaTeX u}

Nový odstavec vytvoříte prázdným řádkem.

Velká většina příkazů se v~TeXu uvozuje zpětným lomítkem (\texinline|\|) povinné
argumenty se píší do složených závorek (\texinline|{}|), nepovinné argumenty do
závorek hranatých (\texinline|[]|). Například text do uvozovek dáte
pomocí~\texinline|\uv{}|, část textu zvýrazníte pomocí~\texinline|\emph{}|.

Snažte se o~to, aby zdrojový kód byl co nejpřehlednější. Pokud nevíte, jaká je
přesná syntaxe určitého příkazu, Google a~manuál {\LaTeX}u vám poradí. Poznámky,
které se nemají objevit ve výsledném textu, pište na řádek začínající \verb|%|.

\begin{texexample}
Lorem ipsum dolor sit amet, consectetur adipiscing.
\uv{Text v~uvozovkách} tint laoreet. Fusce varius
\textbf{tučně} vitae blandit. Cum sociis tetur
\textit{kurzívou}.

\emph{Zvýrazněný text} Venatibus et magnis,
nascetur ridiculus mus. Vestibulum eu neque
sem. Duis luctus vestibulum ultricies.

Nejsou tři tečky (výpustka) jako tři tečky\dots

\noindent\emph{Nápověda}\quad Takto vypíchneme
jedno slovo z~odstavce (např. v~zadání).

% Toto je poznámka, ve výsledném textu není.

\end{texexample}

\section{Vlnky}

Vlnky (\texinline|~|) vkládáme tam, kde nechceme, aby došlo k~rozdělení na dva
řádky. Patří zejména:
\begin{compactitem}
	\item za jednopísmenná slova (a, i~apod.),
	\item za zkratky,
	\item k~veličinám (před nebo za, dle významu),
	\item k~odkazům,
	\item kolem pomlčky.
\end{compactitem}

\begin{texexample}
	Např.~Bob a~Bobek se pohybují rychlostí~$v$.
	Platí rovnice~\eqref{R27S1U2_rovnice-y}, přičemž
	$v$~je rychlost. Praha~--~Ruzyně.
\end{texexample}

Pro doplnění vlnek na některá místa je možné použít skript \texttt{vlnka.sh}
z~repozitáře \texttt{fks-scripts}. Funguje pro češtinu a~slovenštinu a~vlnky
doplňuje zejména na základě slovníku slov, u~kterých vlnka patří. To však
znamená, že může doplnit vlnky i~na místa, kde nejsou nutné, ale také tam, kde
nepatří (do matematiky). Naopak nedoplní vlnky všude. Proto je třeba po jeho
spuštění soubor znovu projít a~smazat přebytečné a~přidat nedoplněné vlnky.
Vhodné je před doplněním vlnek udělat commit, snadno tak zjistíte, kam byly vlnky
doplněny a~můžete přebytečné opravit. Pokud mažete vlnku z~matematiky,
nevypisujte místo ní mezeru, aby se při dalším spuštění skriptu vlnka opět nepřidala.

\section{Nadpisy}

V~textu seriálu používejte nadpis \texinline|\subsection|, ve vzorových řešeních
\texinline|\subsubsection|. Více byste vzorová řešení dělit neměli. Většinu
vzorových řešení není třeba dělit vůbec, naopak u~experimentálních úloh je
dělení většinou nutné (např.~teorie, měření, nejistoty, diskuse, závěr).

\section{Matematické vzorce}
\begin{compactitem}
	\item Inline matematiku píšeme mezi dolary (např.~\texinline|rychlost~$v_1 =
		w_0$|). V~inline matematice nemají být zlomky zapsány jako zlomky, ale pomocí
		lomítka~\texinline|/| (např.~\texinline|rychlost $v=s/t$|).
	\item Jednořádkovou blokovou matematiku píšeme mezi \texinline|\eq{}|
		(ekvivalent \texinline|equation|) a~víceřádkovou mezi \texinline|\eq[m]{}|
		(obdoba \texinline|align|) a~před znaky, na něž chceme zarovnat, dáme
		\texinline|&|. Blokovou matematiku odsazujeme tabulátorem pro lepší čitelnost.
	\item Odkazy děláme pomocí \texinline|\lbl{ID}| na konkrétní řádek.
	\item Vzorec se chová jako část textu (čárku, tečku na konci řádku oddělujeme
		mezerou~\texinline|\,|).
	\item Ve vzorcích používejte normálně znaménka plus, mínus; znaménko krát není
		vhodné psát.
	\item Horní a~dolní index vytvoříte pomocí stříšky (\texinline|^|) nebo
		podtržítka~(\texinline|_|). Pokud je v~indexu více znaků, dejte je do složených
		závorek. Index, který vznikl z~nějakého slova, se píše stojatý~--~před
		podtržítko dejte zpětné lomítko~(\texinline|\_|). Výjimkou je tíhová
		 síla~$F_G$.
	\item Čárku nad veličinou pište pomocí anglického apostrofu (\texinline|'|).
	\item Pokud se v~závorce vyskytuje zlomek, indexy apod., použijeme zvětšující se
		závorky
\end{compactitem}

\begin{texexampletab}
	\texinline|$\%$| & $\%$ &
		procento\\\hline
	\texinline|$x^n, x^{2n}, x_2, x_{22}$| & $x^n, x^{2n}, x_2, x_{22}$ &
		horní a~dolní index (více znaků v~závorkách)\\\hline
	\texinline|$p\_{voda}$| & $p\_{voda}$ &
		dolní indexy, co nejsou veličina (kromě tíhového zrychlení) \\\hline
	\texinline|$\frac{a}{b}, a/b$| & $\frac{a}{b}, a/b$ &
		zlomek, zlomek v~textu\\\hline
	\texinline|$\cdot, \dots, \cdots$| & $\cdot, \dots, \cdots$ &
		tečka mezi dvěma zlomky (spíše nepoužívat), tři tečky (a~tak dále), centrované tři
		tečky (např. v~matici)\\\hline
	\texinline|$\sqrt{x}, \sqrt[n]{x}$| & $\sqrt{x}, \sqrt[n]{x}$ &
		odmocnina, $n$-tá odmocnina\\\hline
	\texinline|$\alpha, \beta, \gamma, \delta$| & $\alpha, \beta, \gamma, \delta$ &
		malá řecká písmena (jiná obdobně)\\\hline
	\texinline|$\Delta, \Gamma$| & $\Delta, \Gamma$ &
		velká řecká písmena (jiná obdobně)\\\hline
	\texinline|$\sin{x}, \cos{x}$| & $\sin{x}, \cos{x}$ &
		sinus, kosinus (jiné obdobně)\\\hline
	\texinline|$\tg{x} = \tan{x}$| & $\tg{x} = \tan{x}$ &
		tangens česky\\\hline
	\texinline|$\sgn{x}$| & $\sgn{x}$ &
		funkce signum \\\hline
	\texinline|$x', x''$| & $x', x''$ &
		čárka nad veličinou \\\hline
	\texinline|$\dot{x}, \ddot{x}$| & $\dot{x}, \ddot{x}$ &
		tečka nad znakem (např. časová derivace), dvě tečky nad znakem (např. druhá
		časová derivace)\\\hline
	\texinline|$\sum_{i=1}^N, \int_a^b$| & $\sum_{i=1}^N, \int_a^b$ &
		suma, integrál\\\hline
	\texinline|$\tsum_{i=1}^N$| & $\tsum_{i=1}^N$ &
		suma v~řádku \\\hline
	\texinline|$\tprod_{i=1}^N$| & $\tprod_{i=1}^N$ &
		součin v~řádku \\\hline
	\texinline|$\tint_a^b \d x$| & $\tint_a^b \d x$ &
		integrál v~řádku, direfenciál \\\hline
	\texinline|$\ztoho$| & $\ztoho$ &
		implikace \\\hline
	\texinline|$\const$| & $\const$ &
		konstanta \\\hline
	\texinline|$\Re z, \Im z$| & $\Re z, \Im z$ &
		reálná a~imaginární část \\\hline
	\texinline|$\eu^{\im x}$| & $\eu^{\im x}$ &
		Eulerovo číslo, imaginární jednotka \\\hline
	\texinline|$\der{f}{x}, \dder{f}{x},|\ \texinline|\pder{f}{x}, \ppder{f}{x}$| &
		$\der{f}{x}, \dder{f}{x}, \pder{f}{x}, \ppder{f}{x}$ &
		derivace (první, druhá, parciální, druhá parciální)\\\hline
	\texinline|$\Laplace, \Dalembert$| & $\Laplace, \Dalembert$ &
		Laplaceův, Dalambertův opetátor\\\hline
	\texinline|$\rot, \Div, \grad$| & $\rot, \Div, \grad$ &
		rotace, divergence, gradient\\\hline
	\texinline|$\op{dup}$| & $\op{dup}$ &
		vlastní (jinak nedefinovaný) operátor \\\hline
	\texinline|$\prumer$| & $\prumer$ &
		průměr\\\hline
	\texinline|$\eqdef,\doteq,\approx,\propto$| & $\eqdef, \doteq, \approx, \propto $ &
		definice, zaokrouhlení, aproximace, úměra\\\hline
	\texinline|$\qed$| & $\qed$ &
		konec důkazu\\\hline
	\texinline|$\vect{F}\_G$| & $\vect{F}\_G$ &
		vektor vysázený tučně\\\hline
	\texinline|$\vectgr{\alpha}$| & $\vectgr{\alpha}$ &
		řecký vektor vysázený tučně\\\hline
	\texinline|$\bod{AB}$| & $\bod{AB}$ &
		geometrický bod\\\hline
	\texinline|$\bb{R}, \bb{C}, \bb{N}$| & $\bb{R}, \bb{C}, \bb{N}$ &
		množiny\\\hline
	\texinline|$\ointo{1}{2}$, $\ointc{1}{2}$, $\cinto{1}{2}$, $\cintc{1}{2}$| &
		$\ointo{1}{2}$, $\ointc{1}{2}$, $\cinto{1}{2}$, $\cintc{1}{2}$ &
 		intervaly\\\hline
	\texinline|$\varpi, \pi$| & $ \varpi, \pi$ &
		rovina, svisle konstanta~$\approx "3.14"$\\\hline
	\texinline|$\dg$| & $\dg$ &
		úhlový stupeň\\\hline
	\texinline|$\C$| & $\C$ &
		stupně Celsia\\\hline
	\texinline|$\ohm$| & $\ohm$ &
		ohm (jednotka)\\\hline
\end{texexampletab}

\begin{texexample}
Rychlost míče~$v\_m=s/t$, ale rychlost míče
\eq{
	v\_m=\frac{s}{t} \,.
}
Tíhová síla~$F_G=ma$, ale gravitační síla
\eq{
	F\_g = \frac{G m_1 m_2}{r^2} \,.
}
Dále platí
\eq{
	y = ax + b \,, \lbl{R27S1U2_rovnice-y}
}
kde $a$~je směrnice a~$b$~posunutí.

Dle rovnice~\eqref{R27S1U2_rovnice-y} něco platí
a~něco zajímavého z~toho určitě plyne. Řešme
soustavu rovnic
\eq[m]{
	x &= a+b+1 \,, \\
	y &= a-b \,.
}
Z~druhé vyjádříme~$b$ a~dosadíme do první.

Pokud je to třeba, použijte zvětšující se závorky
\eq{
	x = \left\lbrace \left[ \( \frac{x^2}{2} + 1 \)
	+ 1 \right] + 1 \right\rbrace \,.
}
Lorentzova síla
\eq{
	\vect F=q\(\vect E+\vect v\times\vect B\) \,.
}
Stopa matice
\eq{
	\tr A_{m,n} = \tr
	\begin{pmatrix}
		a_{1,1} & a_{1,2} & \cdots & a_{1,n} \\
		a_{2,1} & a_{2,2} & \cdots & a_{2,n} \\
		\vdots & \vdots & \ddots & \vdots \\
		a_{m,1} & a_{m,2} & \cdots & a_{m,n}
	\end{pmatrix}
	= \tsum_{j=1}^n a_{j,j} \,.
}
\end{texexample}

\section{Čísla a~jednotky}

Chceme-li napsat desetinné číslo, příp. číslo s~jednotkami, použijeme
uvozovkového pomocníka. Ten je citlivý na mezeru před jednotkou. Pokud je
jednotka součinem několika jiných, píšeme tečku. Čísla do uvozovek píšeme
s~desetinnou tečkou (snadno pak naformátujeme i~výstup z~Gnuplotu). Exponent
desítky napíšeme pomocí \texinline|e|. Čísla oddělujeme do skupin po třech
mezerou \texinline|\,|. Chceme-li psát samostatně jednotku, využijeme příkaz
\texinline|\jd{}|.

\begin{texexample}
Teplota~$"21.3 \C"$, \\
úhel je~$"123\dg"$, \\
$\sigma = "5.67e-8 J.s^{-1}.m^{-2}.K^{-4}"$, \\
$\varphi = "(0.768\,7\pm 0.000\,3) arcsec"$, \\
vzestup o~$"13.2 \%"$, \\
délka~$\lambda = "600 nm" = "0.6 \micro m"$, \\
odpor~$R = "1200 k\ohm"$, \\
rychlost v~$\jd{m.s^{-1}}$,\\
konstanta~$\pi \doteq "3.141\,592\,65"$,\\
konstanta~$\eu \doteq "2.718"$.
\end{texexample}

\section{Chemické vzorce}

Při sazbě různých chemických vzorců a~reakcí si můžeme ušetřit práci balíkem
\texttt{mhchem}, používáme příkaz \texinline|\ce|.

\begin{texexample}
Vzorec látky je~\ce{NO3-}, půlka~\ce{1/2H2O},
izotop~\ce{^{227}_{90}Th+}. Chemická reakce
\eq{
	\ce{CO2 + C -> 2CO} \,. \lbl{R27S1U2_reakce}
}
Odkážeme se na rovnici~\eqref{R27S1U2_reakce}.
Můžeme i~nakreslit vazby,
např.:~\eq{
	\ce{A\bond{-}B\bond{=}C\bond{#}D} \,.
}
Objem vody je~$V_{\ce{H2O}}="5.1 m^3"$.
\end{texexample}

\section{Obrázky}

Obrázky jsou rozděleny do tří kategorií: grafy, datové/velké ilustrační a~drobné
ilustrační (obtékané textem).

\tex|\plotfig{cesta k~souboru .tex}{popiska}{referenční ID} % Gnuplot term eps latex|
\tex|\fullfig{cesta k~souboru}{popiska}{referenční ID}|
\tex|\illfig{cesta k~souboru}{popiska}{referenční ID}{výška v~řádcích}|

Jediným povinným argumentem je cesta k~souboru (implicitně se prohledávají
adresáře \verb|graphics|). Ostatní argumenty lze též zadat jako prázdné tokenem
\texinline|{}|. Chceme-li upravit velikost, popř.~jiný argument velkého obrázku,
napíšeme jej do nepovinného prvního argumentu.

\tex|\fullfig[width=0.6\textwidth]{cesta k~souboru}{popiska}{referenční ID}|

Obrázky (i~zdrojáky) k~úlohám se ukládají do adresáře \verb|problems/graphics|.
Ostatní obrázky (seriál, úvodníček) se ukládají k~dané sérii
(\verb|batchB/graphics|).

\begin{figure}[h!]
\begin{texexamplex}
\begin{texcode}
\illfig{obrazek.eps}{Malý}{R27S1U2_maly}{4}
Obrázek~\ref{R27S1U2_maly} je malinký a~obtéká
ho text. Obrázek~\ref{R27S1U2_velky} je velký.

Lorem ipsum dolor sit amet, consectetur adipiscing
elit. Etiam neque urna, semper quis diam
pellentesque, scelerisque vehicula ante. Maecenas
et feugiat orci. Donec neque eros, eleifend sit amet
adipiscing vel, dictum at arcu. Pellentesque vitae
est ante.



\fullfig{obrazek.eps}{Velký obrázek}{R27S1U2_velky}





Quisque a~urna erat. Suspendisse in est sem. In
hac habitasse platea dictumst. Donec varius,
turpis a~tristique gravida, ligula nunc lobortis.
\end{texcode}
\begin{texview}
	\illfig{obrazek.eps}{Malý}{R27S1U2_maly}{6}
	
	Obrázek~\ref{R27S1U2_maly} je malinký a~obtéká ho text, kdežto
	obrázek~\ref{R27S1U2_velky} je velký.
	
	Lorem ipsum dolor sit amet, consectetur adipiscing elit. Etiam neque urna,
	semper quis diam pellentesque, scelerisque vehicula ante. Maecenas et feugiat
	orci. Donec neque eros, eleifend sit amet adipiscing vel, dictum at arcu.
	Pellentesque vitae est ante.
	
	\begin{center}
		\includegraphics{obrazek.eps}
		\caption{Velký obrázek}
		\label{R27S1U2_velky}
	\end{center}
	
	Quisque a~urna erat. Suspendisse in est sem. In hac habitasse platea dictumst.
	Donec varius, turpis a~tristique gravida, ligula nunc lobortis.
\end{texview}
\end{texexamplex}
\end{figure}

\section{Tabulky}

Nad tabulkou se sází vodorovná čára pomocí \texinline|\toprule|,
\texinline|\midrule| odděluje většinou záhlaví tabulky od zbytku a~pod tabulku
patří vodorovná čára \texinline|\bottomrule|.

Popisek (\texinline|\caption{Popisek}|) se sází nad tabulku. Referenční ID
(\texinline|\label{ID}|) udáváme pro přehlednost a~jednoznačnost ve tvaru
např.~\verb|R27S1U2_id|, kde zaměníme ročník, sérii a~číslo úlohy, \verb|id| je pak
jednoznačný identifikátor v~rámci vzorového řešení jedné úlohy. Příkaz
\texinline|\label| musí být vložen až po \texinline|\caption|. Na tabulku se
odkazujeme pomocí \texinline|\ref{ID}|.

V~záhlaví tabulky můžeme použít příkaz \texinline|\popit|, který funguje stejně
jako \texinline|\popi| (tedy vysází zlomek s~veličinou v~čitateli a~jednotkou ve
jmenovateli), ale je pro použití v~tabulkách, automaticky zarovnává popisek na
střed sloupce (bez dalších argumentů funguje stejně jako
\texinline|\multicolumn{1}{c}{\popi{x}{y}}|). Pomocí nepovinných parametrů je
možné jej rozšířit na více sloupců a~jinak zarovnat, např.~pro dva sloupce
a~zarovnání vlevo použijeme \texinline|\popit[2][l]{t}{s} |.

Čísla v~tabulce by měla být zarovnána většinou podle desetinné čárky,
popř.~vpravo. Záhlaví zarovnejte doprostřed (použitím \texinline|\popit| nebo
\texinline|\multicolumn|, pokud nechcete sázet zlomek s~jednotkou).

\begin{table}[h!]
\begin{texexamplex}
\begin{texcode}
V~tabulce~\ref{R27S1U2_moje-tabulka} je mnoho dat.
Lorem ipsum dolor sit amet, imperdiet metus nunc,
eu faucibus nulla euismod ac. Fusce eu consequat mi.

\begin{table}[h!]
	\centering
	\caption{Moje tabulka}
	\label{R27S1U2_moje-tabulka}
	\begin{tabular}{rrrrrr}
		\toprule
		\multicolumn{1}{c}{$n$} & \popit{d}{cm} &
		\popit{|a'_1|}{m} & \popit{v}{m.s^{-2}} &
		\popit{t_2-t_1}{s} & $\sin{\alpha}$ \\
		\midrule
		1\,113 & 23 & 9 & 3,3 & $"2.1\pm 0.1"$ & 0,11 \\
		1\,233 & 22 & 8 & 4,5 & $"2.1\pm 0.1"$ & 0,17 \\
		1\,313 & 21 & 6 & 4,6 & $"2.1\pm 0.1"$ & 0,21 \\
		\bottomrule
	\end{tabular}
\end{table}

Integer enim nisl, sodales quis porta vel, tincidunt
a~elit. Fusce non ipsum egestas, egestas enim in,
facilisis enim. Vestibulum lacinia viverra est, at.
\end{texcode}
\begin{texview}
	V~tabulce~\ref{R27S1U2_moje-tabulka} je mnoho dat.
	Lorem ipsum dolor sit amet, imperdiet metus nunc,
	eu faucibus nulla euismod ac. Fusce eu consequat mi.
	
	
	\begin{center}
	\caption{Moje tabulka}
	\label{R27S1U2_moje-tabulka}
	\begin{tabular}{rrrrrr}
		\toprule
		\multicolumn{1}{c}{$n$} & \popit{d}{cm} &
		\popit{|a'_1|}{m} & \popit{v}{m.s^{-2}} &
		\popit{t_2-t_1}{s} & $\sin{\alpha}$ \\
		\midrule
		1\,113 & 23 & 9 & 3,3 & $"2.1\pm 0.1"$ & 0,11 \\
		1\,233 & 22 & 8 & 4,5 & $"2.1\pm 0.1"$ & 0,17 \\
		1\,313 & 21 & 6 & 4,6 & $"2.1\pm 0.1"$ & 0,21 \\
		\bottomrule
	\end{tabular}
	\end{center}
	
	\medskip
	
	Integer enim nisl, sodales quis porta vel, tincidunt a~elit. Fusce non ipsum egestas,
	egestas enim in, facilisis enim. Vestibulum lacinia viverra est, at.
\end{texview}
\end{texexamplex}
\end{table}

\section{Seznamy}

Seznam vytvoříte pomocí {\LaTeX}ového prostředí~\texinline|\compactitem|
(nečíslovaný) nebo~\texinline|\compactenum| (číslovaný, druh číslování můžete
uvést v~nepovinné parametru), každou položku označte~\texinline|\item|.

\begin{texexample}
Nákupní seznam s~písmenky:
\begin{compactenum}[a)]
	\item Lorem ipsum dolor sit amet, consectetur adi
		elit. Etiam neque urna, semper quis diam pelent,
		scelerisque vehicula ante.
	\item Maecenas et feugiat orci.
	\item Donec neque eros, eleifend sit amet.
\end{compactenum}

Nákupní seznam s~odrážkami:
\begin{compactitem}
	\item Nulla nec orci imperdiet, dapibus risus vit
		lacinia vestibulum, eleifend purus.
	\item Sed consequat mollis lacinia vestibulum.
\end{compactitem}
\end{texexample}

\section{Poznámky pod čarou}

Poznámky pod čarou se dělají pomocí příkazu~\texinline|\footnote{}|. Číslují se
automaticky a~vysází se v~dolní části strany. Pokud píšeme poznámku pod čarou
uprostřed věty, píše se ihned ke slovu, ke kterému patří. Na konci věty se píše
za tečkou, pokud patří k~celé větě, popř.~před tečkou, vztahuje-li se
k~poslednímu slovu věty. Poznámku nad interpunkcí uděláme pomocí
\texinline|\footnotei{,}{pozn}|.

\begin{texexample}
Lorem ipsum\footnote{Poznámka ke slovu} dolor
sit amet, consectetur adipis cing elit.

Vestibulum nec augue.\footnote{Poznámka
k~celé větě} Pel lentesque convallis ac felis tinci
dunt rhoncus. Cras eleifend tortor sit convallis
amet\footnotei{.}{Poznámka nad interpunkcí}
Morbi rhoncus enim eget.
\end{texexample}

\section{Ostatní}

Na webové odkazy použijte makro \texinline|\url{adresa}|.

Odkazy do textu vložíte pomocí maker~\texinline|\ref{referenční ID}| (obrázky,
tabulky) a~\texinline|\eqref{referenční ID}| (pro rovnice, číslo bude
v~závorkách). Neodkazujte se takto na rovnice, tabulky a~obrázky mimo řešení
příkladu. Číslování je v~různých souborech (brožurka, samotné řešení, ročenka)
různé, takže není jednoznačné.

\begin{texexample}
Více na~\url{http://www.fykos.cz}.
Z~obrázku~\ref{R27S1U2_velky} zjevně neplyne
rovnice~\eqref{R27S1U2_reakce}.
\end{texexample}

\section{Vzorový soubor}

Následující soubor je minimálním příkladem souboru, ve kterém bude fungovat
téměř vše, co je v~tomto dokumentu uvedeno (kromě obrázků, ty je nutno vkládat
běžným způsobem). Můžete tak zde popsané postupy (včetně FYKOSích specialit jako
např.~snadné psaní čísel s~jednotkami pomocí uvozovek) použít i~ve všech
ostatních dokumentech. Překládáme příkazem \texttt{xelatex} (to proto, že FYKOSí
makro s~uvozovkami koliduje s~balíkem \texttt{babel}, který používáme při překladu
pomocí příkazu \texttt{pdflatex}).

\begin{texsource}
\documentclass[10pt,a4paper]{article} % typ dokumentu, velikost písma a strany
\usepackage{amssymb}
\usepackage[intlimits]{amsmath}
\usepackage[no-math]{fontspec}
\usepackage{xltxtra,polyglossia}
\setmainlanguage{czech} % dokument bude v češtině
\mathcode`\,="013B % používá se desetinná čárka
\usepackage{fkssugar} % FYKOS
\usepackage[margin=2cm]{geometry} % okraje stránky
\usepackage{color,graphicx,graphics} % barvičky, obrázky
\usepackage{booktabs,paralist,url} % tabulky, seznamy, odkazy
\usepackage[version=3]{mhchem} % chemie
\newcommand\uv[1]{„#1“} % uvozovky

\author{Tomáš Pikálek}
\title{Můj krásný dokument}
\begin{document}
	Můj první dokument v~\LaTeX u.
\end{document}
\end{texsource}

\end{document}
